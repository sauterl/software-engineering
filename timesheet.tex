%% Allgemeine Definitionen
\documentclass{article} %% Bestimmt die allgemeine Formatierung der Abgabe.
\usepackage{a4wide} %% Papierformat: A4.
\usepackage[utf8]{inputenc} %% Datei wird im UTF-8 Format geschrieben.
%% Unter Windows werden Dateien je nach Editor nicht in diesem Format
%% gespeichert und Umlaute werden dann nicht richtig erkannt.
%% Versucht in diesem Fall "utf8" auf "latin1" (ISO 8859-1) wechseln.
\usepackage[T1]{fontenc} %% Format der Zeichen im erstellten PDF.
\usepackage[english]{babel} %% Regeln für automatische Worttrennung.
\usepackage{fancyhdr} %% Paket um einen Header auf jeder Seite zu erstellen.
\usepackage{lastpage} %% Wird für "Seite X von Y" im Header benötigt.
                      %% Damit das funktioniert, muss pdflatex zweimal
                      %% aufgerufen werden.
\usepackage{enumerate} %% Hiermit kann der Stil der Aufzählungen
                       %% verändert werden (siehe unten).


\usepackage{listings}

\usepackage{amssymb} %% Definitionen für mathematische Symbole.
\usepackage{amsmath} %% Definitionen für mathematische Symbole.

\usepackage{wasysym} % %For lightning bolt

\usepackage{tikz}  %% Paket für Grafiken (Graphen, Automaten, etc.)
\usetikzlibrary{automata} %% Tikz-Bibliothek für Automaten
\usetikzlibrary{arrows}   %% Tikz-Bibliothek für Pfeilspitzen


%% Linke Seite des Headers
\lhead{\course\\Fall 2015}
%% Rechte Seite des Headers
\rhead{\mg, \sh, \ej, \ls\\ Page \thepage\ of \pageref{LastPage}}
%% Höhe des Headers
\usepackage[headheight=36pt]{geometry}
%% Seitenstil, der den Header verwendet.
\pagestyle{fancy}

\newcommand{\ls}{Loris Sauter}
\newcommand{\sh}{Silvan Heller}
\newcommand{\mg}{Max Grüner}
\newcommand{\ej}{Eddie Joseph}
\newcommand{\semester}{Herbstsemester 2015}
\newcommand{\course}{Software Engineering}
\newcommand{\homeworkNumber}{2}


\begin{document}
\part{Time Sheet}
\section{Planning}
For the API planning we invested about 3 hours meeting.
\section{Coding}
The coding time of the delivered API (package \texttt{api} and sub-packages) is located around 1 hour.
\section{Documentation}
For writing the documentation of the delivered API, about 4-5 hours were spend.
\section{Review}
\subsection{Code}
The code review took not too long, just 0.5 hours.
\subsection{Documentation}
The review of the documentation was split into individual and general review during a meeting. All in all somewhere between 1 and 2 hours of documentation review we had.
\section{Total}
The delivered API (package \texttt{api} and sub-packages) the working time is build up as followed:
\begin{center}
	\begin{tabular}{|l|c|}
	\hline  Task & Time (in hours) \\ 
	\hline  Planning &  3\\ 
	\hline  Coding &  1\\ 
	\hline  Documentation & 5 \\ 
	\hline  Review - Code&  0.5\\ 
	\hline  Review - Doc &  2\\ 
	\hline\hline   Total &  11.5\\ 
	\hline 
\end{tabular} 
\end{center}
\end{document}
