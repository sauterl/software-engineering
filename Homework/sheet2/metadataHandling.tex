%% Allgemeine Definitionen
\documentclass{article} %% Bestimmt die allgemeine Formatierung der Abgabe.
\usepackage{a4wide} %% Papierformat: A4.
\usepackage[utf8]{inputenc} %% Datei wird im UTF-8 Format geschrieben.
%% Unter Windows werden Dateien je nach Editor nicht in diesem Format
%% gespeichert und Umlaute werden dann nicht richtig erkannt.
%% Versucht in diesem Fall "utf8" auf "latin1" (ISO 8859-1) wechseln.
\usepackage[T1]{fontenc} %% Format der Zeichen im erstellten PDF.
\usepackage[english]{babel} %% Regeln für automatische Worttrennung.
\usepackage{fancyhdr} %% Paket um einen Header auf jeder Seite zu erstellen.
\usepackage{lastpage} %% Wird für "Seite X von Y" im Header benötigt.
                      %% Damit das funktioniert, muss pdflatex zweimal
                      %% aufgerufen werden.
\usepackage{enumerate} %% Hiermit kann der Stil der Aufzählungen
                       %% verändert werden (siehe unten).


\usepackage{listings}

\usepackage{amssymb} %% Definitionen für mathematische Symbole.
\usepackage{amsmath} %% Definitionen für mathematische Symbole.

\usepackage{wasysym} % %For lightning bolt

\usepackage{tikz}  %% Paket für Grafiken (Graphen, Automaten, etc.)
\usetikzlibrary{automata} %% Tikz-Bibliothek für Automaten
\usetikzlibrary{arrows}   %% Tikz-Bibliothek für Pfeilspitzen


%% Linke Seite des Headers
\lhead{\course\\Fall 2015}
%% Rechte Seite des Headers
\rhead{\mg, \sh, \ej, \ls\\ Page \thepage\ of \pageref{LastPage}}
%% Höhe des Headers
\usepackage[headheight=36pt]{geometry}
%% Seitenstil, der den Header verwendet.
\pagestyle{fancy}

\newcommand{\ls}{Loris Sauter}
\newcommand{\sh}{Silvan Heller}
\newcommand{\mg}{Max Grüner}
\newcommand{\ej}{Eddie Joseph}
\newcommand{\semester}{Herbstsemester 2015}
\newcommand{\course}{Software Engineering}
\newcommand{\homeworkNumber}{2}


\begin{document}
\part{Meta Data Handling}
\section{Goal}
The goal is to store meta data efficient and easy accessible within the repository.
\section{Requirements}
To achieve the mentioned goal the following requirements must be fitted:
\begin{itemize}
	\item New meta data must get added easily,
	\item Meta data must be searchable,
	\item Meta data has several categories (name, timestamp, description ...)
	\item Meta data must be easy readable
	\item Meta data and data set must be linked
	\item Meta data is structured.
\end{itemize}
\section{Solution}
Due to the above listed requirements, we came up with the solution to use a JSON-file to store the meta data.
To have easy access for searches through the repository and its data set's meta data, we decided to use one single
file to store the meta data in. This centralised spot containing the meta data has the following structure:
\begin{lstlisting}
{
    "repository":{
		"version":"0.1",
		"timestampe:"2014-09-18T13:40:18",
		"datasets":
		[
			{
				"id":1,
				"name":"big folder",
				"timestamp":"2014-09-18T13:41:35",
				"description":"Some files",
				"filecount":35,
				"size":9369436,
				"filename":"myWork",
				"filetype":"directory"
			}
		]
	}
}
\end{lstlisting}
This meta data JSON-file is located in the root of a repository.

\noindent For each data set that is added to the repository, an entry in the meta data file is created as well as a folder. The folder's name has the format \texttt{DataSet\_ID} where the ID is corresponds wit the ID in the meta data file.
\end{document}
