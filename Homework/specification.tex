%% Allgemeine Definitionen
\documentclass{article} %% Bestimmt die allgemeine Formatierung der Abgabe.
\usepackage{a4wide} %% Papierformat: A4.
\usepackage[utf8]{inputenc} %% Datei wird im UTF-8 Format geschrieben.
%% Unter Windows werden Dateien je nach Editor nicht in diesem Format
%% gespeichert und Umlaute werden dann nicht richtig erkannt.
%% Versucht in diesem Fall "utf8" auf "latin1" (ISO 8859-1) wechseln.
\usepackage[T1]{fontenc} %% Format der Zeichen im erstellten PDF.
\usepackage[english]{babel} %% Regeln für automatische Worttrennung.
\usepackage{fancyhdr} %% Paket um einen Header auf jeder Seite zu erstellen.
\usepackage{lastpage} %% Wird für "Seite X von Y" im Header benötigt.
                      %% Damit das funktioniert, muss pdflatex zweimal
                      %% aufgerufen werden.
\usepackage{enumerate} %% Hiermit kann der Stil der Aufzählungen
                       %% verändert werden (siehe unten).

\usepackage{amssymb} %% Definitionen für mathematische Symbole.
\usepackage{amsmath} %% Definitionen für mathematische Symbole.

\usepackage{wasysym} % %For lightning bolt

\usepackage{tikz}  %% Paket für Grafiken (Graphen, Automaten, etc.)
\usetikzlibrary{automata} %% Tikz-Bibliothek für Automaten
\usetikzlibrary{arrows}   %% Tikz-Bibliothek für Pfeilspitzen


%% Linke Seite des Headers
\lhead{\course\\Fall 2015}
%% Rechte Seite des Headers
\rhead{\mg, \sh, \ej, \ls\\ Page \thepage\ of \pageref{LastPage}}
%% Höhe des Headers
\usepackage[headheight=36pt]{geometry}
%% Seitenstil, der den Header verwendet.
\pagestyle{fancy}

\newcommand{\ls}{Loris Sauter}
\newcommand{\sh}{Silvan Heller}
\newcommand{\mg}{Max Grüner}
\newcommand{\ej}{Eddie Joseph}
\newcommand{\semester}{Herbstsemester 2015}
\newcommand{\course}{Software Engineering}
\newcommand{\homeworkNumber}{2}


\begin{document}
\part*{Command specifications}
	% % Trying to get a man-pages style (like git)
	% % specification template:
%	\section*{COMMAND}
%	\subsection*{Name}
%	\texttt{COMMAND} - SHORT
%	\subsection*{Synopsis}
%	\texttt{\textbf{COMMAND} OPTIONS PARAMETERS}
%	\subsection*{Description}
%	DESCRIPTION OF THE COMMAND\\
%	
%	\noindent On success, ...
%	\subsubsection*{Options}
%	\begin{description}
%		\item[\texttt{OPTION}] DESCRIPTION
%		\item[\texttt{OPTION}] DESCRIPTION
%	\end{description}
%	
%	\subsubsection*{Parameters}
%	\begin{description}
%		\item[\texttt{PARAM}] DESCRIPTION
%		\item[\texttt{PARAM}] DESCRIPTION
%	\end{description}
%	\subsubsection*{Exit status}
%	\begin{tabular}{ll}
%		0 &  if OK,\\ 
%		1 &  if command-parameter specific error occurred\\
%		2 &  if the command could not be executed\\
%		4 &  if any other error occurred\\
%	\end{tabular}
%	\newpage
	% % END OF TEMPLATE
		\section*{ADD}
		\subsection*{Name}
		\texttt{add} - Add data set to the repository
		\subsection*{Synopsis}
		\texttt{\textbf{add} [-v] <name> [<description>]}
		\subsection*{Description}
		Add the specified data set with path \texttt{<name>} and optional \texttt{description} to the repository.\\
		
		\noindent
		\subsubsection*{Options}
		\begin{description}
			\item[\texttt{-v}] verbose progress
		\end{description}
		
		\subsubsection*{Parameters}
		\begin{description}
			\item[\texttt{<name>}] Path of the data set.
			\item[\texttt{<description>}] \textbf{Optional} A description of the data set
		\end{description}
		
		\subsection*{Output}
		The meta data of the data set added, tab separated. Format:\\<id> \textbackslash t <name> \textbackslash t <timestamp> \textbackslash t <description> \textbackslash t <number of files> \textbackslash t <size>\\
		In case of abort a detailed error message
		
		\subsubsection*{Exit status}
		\begin{tabular}{ll}
			0 &  if successful,\\ 
			1 &  if the file/folder under the given \texttt{<name>} does not exist\\ 
			2 &  if the data set could not be added to the repository\\
			4 &  if any other error occurred\\
		\end{tabular}
		
		\subsection*{Time Estimation}
		15 hours of work, in particular testing Java I/O with large data sets,testing various operating systems and privilege levels (admin vs non-admin), building methods for creating meta data and creating the structure of the application. \\
		\noindent Since all of us have recent working experience in java, we add on top of that an estimated 2 hours per person to get our work flow set up.
		\newpage
		\section*{REMOVE}
		\subsection*{Name}
		\texttt{remove} - remove a data set
		\subsection*{Synopsis}
		\texttt{\textbf{remove} [-v] <name>}
		\subsection*{Description}
		Remove the data set with the path specified by \texttt{<name>}\\
		
		\noindent
		\subsubsection*{Options}
		\begin{description}
			\item[\texttt{-v}] verbose progress
		\end{description}
		
		\subsubsection*{Parameters}
		\begin{description}
			\item[\texttt{<name>}] The path of the data set (e.g. the name of the file/folder)
		\end{description}
		\subsection*{Output}
		No output, if the operation ended successfully. In case of abort a detailed error message.
		\subsubsection*{Exit status}
		\begin{tabular}{ll}
			0 &  if successful,\\ 
			1 &  if the data set with the specified \texttt{<name>} does not exist\\ 
			2 &  if the data set could not be removed\\
			4 &  if any other error occurred\\
		\end{tabular}
				\subsection*{Time Estimation}
				3 hours of work \\
				\noindent
		\newpage
		\section*{COPY}
		\subsection*{Name}
		\texttt{copy} - copies a data set from the repository to any location in the file system.
		\subsection*{Synopsis}
		\texttt{\textbf{copy} [-v] <name> <location>}
		\subsection*{Description}
		Copies the data set with path \texttt{<name>} to the given \texttt{<location>}.
		If the operation succeeded, the data set remains in the repository and is accessible with the original name at  \texttt{<location>}.\\
		
		\noindent
		\subsubsection*{Options}
		\begin{description}
			\item[\texttt{-v}] verbose progress
		\end{description}
		
		\subsubsection*{Parameters}
		\begin{description}
			\item[\texttt{<name>}]  Path of the data set.
			\item[\texttt{<location>}] Path of the new location for the data set.
		\end{description}
		\subsection*{Output}
		No output, if the operation ended successfully. In case of abort detailed error message.
		\subsubsection*{Exit status}
		\begin{tabular}{ll}
			0 &  if successful,\\ 
			1 &  if the data set with the specified \texttt{<id>} does not exist or the \texttt{<location>} parameter is invalid\\
			2 &  if the data set could not be copied (e.g. if  a file / folder with given name already exists at \texttt{<location>})\\
			4 &  if any other error occurred\\
		\end{tabular}
		
		\subsection*{Time Estimation}
		2 hours of work if done after the add and remove commands, since they are quite similar in syntax and copy is very similar to the add command.\\
		\noindent
		
		\newpage
		\section*{REPLACE}
		\subsection*{Name}
		\texttt{replace} - replaces a data set with another one.
		\subsection*{Synopsis}
		\texttt{\textbf{replace} [-v] <original> <new>}
		\subsection*{Description}
		Replaces the data set at path \texttt{<original>} with the data set located at \texttt{<new>}.\\
		
		\noindent
		\subsubsection*{Options}
		\begin{description}
			\item[\texttt{-v}] verbose progress
		\end{description}
		
		\subsubsection*{Parameters}
		\begin{description}
			\item[\texttt{<original>}] The path of the data set that should be replaced.
			\item[\texttt{<new>}] The path to the data set replacing the original one
		\end{description}
		\subsection*{Output}
		The new meta data of the data set replaced, tab separated. Format:\\<id> \textbackslash t <name> \textbackslash t <timestamp> \textbackslash t <description> \textbackslash t <number of files> \textbackslash t <size>\\
		In case of abort a detailed error message
		\subsubsection*{Exit status}
		\begin{tabular}{ll}
			0 &  if successful,\\ 
			1 &  if command-parameter specific error occurred\\
			2 &  if the data set with specified \texttt{<id>} does not exist or the \texttt{<path>} parameter is invalid\\
			4 &  if any other error occurred\\
		\end{tabular}
				\subsection*{Time Estimation}
				2 hours of work. Very similar to the copy command\\
				\noindent
		\newpage
		\section*{LIST}
		\subsection*{Name}
		\texttt{list} - lists all data set in the repository and their meta data
		\subsection*{Synopsis}
		\texttt{\textbf{list} [-v]}
		\subsection*{Description}
		Lists all data sets and their meta data.\\
		
		%\noindent On success, a list of all data sets gets printed. Additionally to the name of the data set, its description %(if one exists), size and number of files within get printed.
		\subsubsection*{Options}
		\begin{description}
			\item[\texttt{-v}] verbose progress
		\end{description}
		
%		\subsubsection*{Parameters}
%		\begin{description}
%			\item[\texttt{PARAM}] DESCRIPTION
%			\item[\texttt{PARAM}] DESCRIPTION
%		\end{description}
		\subsection*{Output}
		A list of all data sets in the repository. Separated with tabs. Format:\\<id> \textbackslash t <name> \textbackslash t <timestamp> \textbackslash t <description> \textbackslash t <number of files> \textbackslash t <size> \textbackslash n \\<id> \textbackslash t <name> \textbackslash t <timestamp> \textbackslash t <description> \textbackslash t <number of files> \textbackslash t <size>\\
		Or a detailed error message in case of abort.

		\subsubsection*{Exit status}
		\begin{tabular}{ll}
			0 &  if successful,\\ 
%			1 &  if the current working directory is not a repository\\
%			2 &  if the command could not be executed\\
			4 &  if any error occurred\\
		\end{tabular}
				\subsection*{Time Estimation}
				6 hours of work, mainly for traversing the metadata-storage and testing the behaviour of \texttt{<list>} with various commands \\
				\noindent
		\newpage
		
		\section*{INIT}
		\subsection*{Name}
		\texttt{init} - initializes the repository in the current working directory
		\subsection*{Synopsis}
		\texttt{\textbf{init} [-v]}
		\subsection*{Description}
		Initializes the repository in the current working directory.
		During the initialization files for storing meta data are generated.\\
		
		\noindent
		\subsubsection*{Options}
		\begin{description}
			\item[\texttt{-v}] verbose progress
		\end{description}
		
%		\subsubsection*{Parameters}
%		\begin{description}
%			\item[\texttt{PARAM}] DESCRIPTION
%			\item[\texttt{PARAM}] DESCRIPTION
%		\end{description}
		\subsection*{Output}
		No output if the operation ended successfully. In case of abort a detailed error message.
		\subsubsection*{Exit status}
		\begin{tabular}{ll}
			0 &  if successful,\\ 
%			1 &  if command-parameter specific error occurred\\
%			2 &  if the command could not be executed\\
			4 &  if any error occurred\\
		\end{tabular}
				\subsection*{Time Estimation}
				3 hours of work since I/O testing on various operating systems and privilege levels should have already been done with the add command \\
				\noindent
		\newpage
		
		\section*{SEARCH}
		\subsection*{Name}
		\texttt{search} - searches for one or more data sets
		\subsection*{Synopsis}
		\texttt{\textbf{search} [-v][-a | -o] [-name=<pattern>] [-description=<pattern>] [-id=<id>] [-timestamp=<time>] [-size=<size>] [-files=<amount>]}
		\subsection*{Description}
		Searches for data sets based on their meta data.
		\noindent
		\subsubsection*{Options}
		\begin{description}
			\item[\texttt{-v}] verbose progress
			\item[\texttt{-a}] match all given search parameters
			\item[\texttt{-o}] match one of the given search parameters
		\end{description}
		\subsubsection*{Parameters}
		\begin{description}
			\item[\texttt{-name=<pattern>}] The pattern the path should match
			\item[\texttt{-description=<pattern>}] The pattern the description should match
			\item[\texttt{-id=<id>}] ID of the data set
			\item[\texttt{-timestamp=<time>}] Timestamp of the data set
			\item[\texttt{-size=<size>}] The size in Bytes of the data set
			\item[\texttt{-files=<amount>}] The amount of files the data set contains
		\end{description}
		\subsection*{output}
		A list of all data sets in the repository matching the search pattern, separated with tabs. Format:\\<id> \textbackslash t <name> \textbackslash t <timestamp> \textbackslash t <description> \textbackslash t <number of files> \textbackslash t <size> \textbackslash n \\<id> \textbackslash t <name> \textbackslash t <timestamp> \textbackslash t <description> \textbackslash t <number of files> \textbackslash t <size>\\
		No output if none of the data sets matches the pattern, error message in case of abort.
		\subsubsection*{Exit status}
		\begin{tabular}{ll}
			0 &  if successful,\\ 
			1 &  if a command-parameter specific error occurred\\
			2 &  if the command could not be executed\\
			4 &  if any other error occurred\\
		\end{tabular}
				\subsection*{Time Estimation}
				6 hours of work, in particular for testing parameters\\
				\noindent
		\newpage

\end{document}
